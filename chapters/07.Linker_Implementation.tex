\chapter{Linker Implementation}
The Lucy compiler aims to compile the Lucy code into Lucy X module with solving the local linkage. The \texttt{Import} keyword is reserved for global linkage. The global linkage is solved during linking time by the Lucy linker. This chapter first discusses how the routine and module are represented and how the global linking is solved for those global function calls. Then we discuss the detailed implementation of the Lucy linker.


\section{Subroutine Representation}
In Lucy X assembly language, functions are represented as subroutines. A subroutine provides scope isolation and stack memory for parameter passing and function execution. For each subroutine produced by the Lucy compiler, it will be prefixed with the module name to differentiate the subroutines.
\begin{lstlisting}
package "string"

func len(s: string) {}
\end{lstlisting}
will be translated to
\begin{lstlisting}[language=LucyX]
string:len {
  pop s
}
\end{lstlisting}
In this case, inside a package, even the two functions are in different files, they cannot be the same name. Otherwise when the global linker links all module files into one module, the naming will be duplicated and an \texttt{OverdefinedException} will be thrown. But if these two functions are in two different packages, it will be fine if they have the same name since they will be prefixed their package's name.

\section{Module Representation}
The linker will read in different module files and finally produce one final module. If the module files are some partial files of the package, the linker will merge them into one module and if the one of the module files is a main module, the linker will merge them and produce a new main module. Note that if the modules coming from different packages and there is no main module in those files, the linker will generate an exception, i.e. linker cannot link two modules from different packages except one of them is main module.
\begin{lstlisting}[language=LucyX]
; file 1
; module name: string
string:len{}

; file 2
; module name: string
string:charAt{}

; by default, it will be linked and merged into file string
string:len{}
string:charAt{}
\end{lstlisting}
The above example shows that to link a module, the linker simply merges all modules files into one. If the subroutine is already in the module, then an \texttt{OverdefinedException} will be thrown.
\begin{lstlisting}[language=LucyX]
; string
string:len{}
string:charAt{}

; array
array:len{}
array:remove{}

; main
main{}

; by default, it will merged into file main
string:len{}
string:charAt{}
array:len{}
array:remove{}
main{}
\end{lstlisting}
Linking an executable module is a little different with linking a pacakage. To link a package, the linker just merge all module files into one module and the module name will not be changed. In this case, the package name cannot be different in different module files. To link an executable module, the module files can come from different packakges and the linker will create a main module then all subroutines will be pushed into the main module.

\section{Linker Implementation}
Duis consequat mauris a dolor ultrices eleifend. Quisque nulla est, dictum et augue ut, iaculis aliquet sem. Integer in sem dapibus, suscipit est eu, pellentesque velit. Praesent lacus massa, scelerisque ac rutrum nec, sodales nec risus. Aenean sed rhoncus nunc. Ut eget elit sapien. Aenean iaculis velit vel porta consequat. Ut eros ex, rutrum dapibus ex varius, gravida condimentum sapien. Etiam finibus, nulla sed sodales tincidunt, nulla purus malesuada nunc, at imperdiet eros lorem ut dui. In ullamcorper eget eros in euismod. Donec ipsum mi, viverra id urna et, iaculis aliquam tellus. Duis lobortis pellentesque elit, ut tempus mauris pretium sollicitudin.

Duis id facilisis justo. Suspendisse sed scelerisque felis. Donec varius convallis tempor. Praesent ipsum dui, molestie non molestie eu, tincidunt at enim. Interdum et malesuada fames ac ante ipsum primis in faucibus. Fusce non tempor leo. Praesent sit amet hendrerit purus. Aliquam ut tortor velit. Praesent sit amet dui euismod, semper sem at, viverra elit. Cras rutrum aliquet urna ac malesuada. Duis dui sapien, rutrum at congue et, iaculis eget lectus. Nulla lectus massa, ultrices ac tempus nec, lacinia quis nisi. Nulla turpis ipsum, faucibus eget nibh vitae, venenatis lobortis lectus. Maecenas pretium dictum elit. Duis sem risus, laoreet a eros vitae, semper condimentum neque.

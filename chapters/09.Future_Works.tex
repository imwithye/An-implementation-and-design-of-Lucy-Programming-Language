\chapter{Future Works}
Lucy is a very simple language which has only about 20 instructions. This chapter discuss the future work of Lucy language and the plan of solving the current issues. First we will discuss the optimization problem of the Lucy X assembly language. Second, more language features will be discussed. Last but not least, we will discuss the plan of using LLVM to replace the current Lucy virtual machine.


\section{Optimization}
As mentioned before, the Lucy compiler uses a naive way to perform the code transformation and the strategy leads more memory usage and redundant instructions. Instead of changing the compilation strategy, a better solution is to add a Lucy optimizer into the system. The optimizer will build CFG for each statement blocks and perform code analysis on each statement. \\
A simple and efficient optimization technique is the Sparse Conditional Constant Propagation. The Sparse Conditional Constant Propagation pass is an optimization frequently applied in compilers after conversion to static single assignment form (SSA). It simultaneously removes some kinds of dead code and propagates constants throughout a program. Moreover, it can find more constant values, and thus more opportunities for improvement, than separately applying dead code elimination and constant propagation in any order or any number of repetitions. The LLVM's implementation of the SCCP pass is attached at the appendix part.
Moreover, Lucy uses stack to perform the parameter passing. The Dead Argument Removal pass is able to remove all dead parameter inside the function, in which case, the \texttt{pop} instruction can be omitted from the routine. After removing the \texttt{pop} instruction, the subroutine call should also be adjusted. The corresponding \texttt{push} instruction will also be removed from the caller. \\
When linking an executable module, the linker will add all subroutines from the imported package and produce a main module. The Lucy optimizer can be used to remove all subroutines that are not called from the main module. This will reduce the file size of the target executable file though it can not be used to improve the execution speed.


\section{Language Features}
Lucy now provides very basic language features like conditional branch and while loop. A static typed \texttt{struct} can be easily instructed to Lucy without modifying the Lucy X's implementation. To add dynamic typed \texttt{struct} or class oriented programming pattern, new instruction shall be added to the Lucy X language to wrap around the \texttt{struct} or \texttt{class}. In addition, function will be the first-class citizen in Lucy language which means function can be passed into another function as a parameter. Moreover, anonymous function is planed for next version of Lucy language.
In the last chapter, we discussed the error handling in Lucy. Since Google has already discussed the advantages and disadvantages of these two different syntax styles, Lucy is not going to change in recent releases.
A doable language feature is to support linking the native Java code. Since Lucy virtual machine is purely based on Java and the register in Lucy VM is just a Java object which stores a value, a Java object can be stored in the register as well. A cross linking can be done between the Java class and the Lucy code. This feature will make Lucy more powerful and provides tons of libraries.


\section{Native Machine Code with LLVM}
There is no direct relationship between Lucy and Lucy X assembly language. Lucy is just the high level language and it can be compiled, or translated into any other languages. To make Lucy become a language which can be used in production environment, LLVM backend will be used as the backend compiler to generate the native machine code. At that moment, the whole Lucy virtual machine and Lucy X will be replaced with the LLVM backend generator. The Lucy X abstract syntax tree will be directly compiled into LLVM bit code and the LLVM's optimizer and linker will kick in after the compilation. Replacing the Lucy virtual machine with LLVM backend not only brings the performance but also a lot of powerful tools. Lucy could be a system level programming language by cross linking the C libraries. \\
Lucy and Lucy X will be separated into two different projects. Lucy X will keep staying simple and easy to understand while Lucy aims to be a powerful general purpose programming language. 

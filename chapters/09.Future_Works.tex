\chapter{Future Works}
Lucy is a very simple language which has only about 20 instructions. This chapter discuss the future work of Lucy language and the plan of solving the current issues. First we will discuss the optimization problem of the Lucy X assembly language. Second, more language features will be discussed. Last but not least, we will discuss the plan of using LLVM to replace the current Lucy virtual machine.


\section{Optimization}
As mentioned before, the Lucy compiler uses a naive way to perform the code transformation and the strategy leads more memory usage and redundant instructions. Instead of changing the compilation strategy, a better solution is to add a Lucy optimizer into the system. The optimizer will build CFG for each statement blocks and perform code analysis on each statement. \\
A simple and efficient optimization technique is the Sparse Conditional Constant Propagation. The Sparse Conditional Constant Propagation pass is an optimization frequently applied in compilers after conversion to static single assignment form (SSA). It simultaneously removes some kinds of dead code and propagates constants throughout a program. Moreover, it can find more constant values, and thus more opportunities for improvement, than separately applying dead code elimination and constant propagation in any order or any number of repetitions. The LLVM's implementation of the SCCP pass is attached at the appendix part.
Moreover, Lucy uses stack to perform the parameter passing. The Dead Argument Removal pass is able to remove all dead parameter inside the function, in which case, the \texttt{pop} instruction can be omitted from the routine. After removing the \texttt{pop} instruction, the subroutine call should also be adjusted. The corresponding \texttt{push} instruction will also be removed from the caller. \\
When linking an executable module, the linker will add all subroutines from the imported package and produce a main module. The Lucy optimizer can be used to remove all subroutines that are not called from the main module. This will reduce the file size of the target executable file though it can not be used to improve the execution speed.


\section{Language Features}
Duis consequat mauris a dolor ultrices eleifend. Quisque nulla est, dictum et augue ut, iaculis aliquet sem. Integer in sem dapibus, suscipit est eu, pellentesque velit. Praesent lacus massa, scelerisque ac rutrum nec, sodales nec risus. Aenean sed rhoncus nunc. Ut eget elit sapien. Aenean iaculis velit vel porta consequat. Ut eros ex, rutrum dapibus ex varius, gravida condimentum sapien. Etiam finibus, nulla sed sodales tincidunt, nulla purus malesuada nunc, at imperdiet eros lorem ut dui. In ullamcorper eget eros in euismod. Donec ipsum mi, viverra id urna et, iaculis aliquam tellus. Duis lobortis pellentesque elit, ut tempus mauris pretium sollicitudin.

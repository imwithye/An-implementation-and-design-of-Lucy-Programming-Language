%!TEX root = ../dissertation.tex
\begin{savequote}[75mm]
A language that doesn't affect the way you think about programming is not worth knowing.
\qauthor{Alan J. Perlis}
\end{savequote}

\chapter{Lucy Programming Language}
The Lucy programming language is a dynamic type language with static type checking. The goal of Lucy language is to provide an elegant and safe way to ensure the correctness of scripting program. Similar to python, Lucy provides a run time type checking system to ensure each operation is done correctly. Meanwhile, inspired by TypeScript, Lucy provides static type checking to improve the robustness of the Lucy program. And every thing in Lucy is passed by value, even if the parameter is an array. \\
The following sections will first give an overview of the syntaxes and then describe how a Lucy program is organized. We also discuss the benefits and trade-offs of the syntaxes design. In the last part of this chapter, we will discuss how the Lucy compiler and linker works and how this method improve the compilation speed of the Lucy program.


\section{A Tour with Lucy}
Tradition suggests that the first program in a new language should print the words "Hello World!" on the screen. In Lucy Programming Language, it can be done with invoking the \texttt{print} function from the \texttt{std} library.
\begin{lstlisting}
import "std"

func main() {
  std::print("Hello World!\n")
}
\end{lstlisting}
If you have written code in C or C++, this code will be very familiar to you. The Lucy program starts from the main function, which is very similar to C programming language, and it executes the code line by line. Moreover, unlike C and C++, Lucy does not require semicolons at the end of every statement.

Lucy does not requires the main function to return an \texttt{int} number as the program's exiting code and in fact, the virtual machine will detect the return statement during run time and return a none value if the function does not return anything. The detailed description of the function behavior will be introduced in the function section. Though Lucy is a dynamic type programming language, i.e. type of the the variable can be changed during the run time, Lucy still requires compiling and linking to the Lucy X assembly code so that it can be interpreted by Lucy Virtual Machine. The mechanism behind Lucy, Lucy X and its virtual machine will be discussed in the chapters of Lucy X Assembly Language and Lucy Virtual Machine. Lucy aims to provide a way to mix static type and dynamic type. The front-end semantic checker will ensure the correctness of the static type assignment, and the virtual machine, which provides a run time checking system, will ensure the correctness of the dynamic type assignment. The trade-offs of this syntax design and the implementation will be discussed in the syntax design section and the implementation chapter.

\subsection{Simple Value}
Lucy uses the keyword \texttt{var} to create a variable and the initial value of the variable will be set to none if no keyword specifying the type is given.
\begin{lstlisting}
var a              // a is a optional type with none value
var a = 10         // a is a number type with number value 10
var a: number      // a is a number type with number value 0
var a: number = 10 // a is a number type with number value 10
\end{lstlisting}
A comma can be used to separate multiple var declarations and to assign the variable a static type, the keyword representing the type should be present after every single declaration. For example:
\begin{lstlisting}
var a, b: number   // a is optional(underdetermined) type but b is number type
var a: number, b: number      // both a and b are number type
\end{lstlisting}
The default values of the build-in types are set as following:
\begin{itemize}
  \item number: 0
  \item boolean: false
  \item string: ""
  \item array: []
  \item optional: none
\end{itemize}

\subsection{Array}
In Lucy, array is a primitive type, immutable, and passed by value when invoking a function call. Lucy X provides 2 low level instructions, getter and len function, to access array items and its length.
\begin{lstlisting}
var length = len(items)
var value = items[0]
items[0] = value        // invalid, since array is immutable
\end{lstlisting}
Decalaring the item of an array is also required for a static type array variable. An two dimensional arrays can be implemented by an array that contains another one dimensional array. Array also can be represented as a literal value. For example:
\begin{lstlisting}
var numbers1: [number] = []
var numbers2 = [0, 1]
var strings1: [string] = []
var strings2 = ["hello", "world"]
var optionals1: [] = []
var optionals2: [] = [0, "string"]
var twoDimensional: [[number]] = [[0, 1, 2], [3, 4, 5]]
\end{lstlisting}
The + operator is used to concatenate two arrays and the result will be a new array that contains all elements of these two. When copying values to the new array, deep copy strategy is used so that every value inside these two arrays will be copied to the result array. Note that optional type propagation rule is applied here. For example:
\begin{lstlisting}
var result1 = ["hello", 0] + [0, [1, "world"]]
result1 == ["hello", 0, 0, [1, "world"]]
var result2 = [0, 1] + [0, 2]
result2 == [0, 1, 0, 2]  // type of result2 is [number]
\end{lstlisting}

\subsection{Type}
Lucy is a dynamic type programming language, which means the type of the variable can be changed during the runtime. Lucy also provides a static type checking system and introduces the optional type the language. In Lucy, a variable without associating a type keyword will be considered as an optional type value. But if the initial value of the variable has a static type, the type of the variable will be inferred from that static type. The question mark will force the static type to be a optional type. Only the optional type variables can be changed during runtime. For example:
\begin{lstlisting}
var a = 0      // a is a variable of number type
var b = 0?     // b is optional type with value 0
b = "string"
b = false
\end{lstlisting}
Moreover, a variable of optional type can be assigned with any types but a variable of a static type can only be assigned with the value of that type. For example:
\begin{lstlisting}
var a, b: number  // a is optional but b is number
a = b             // valid
b = a             // invalid; type error
\end{lstlisting}
Optional type will propagate to the result of the expression when connecting two different values:
\begin{lstlisting}
var a = 0?, b: number = 0
var c = a + b        // c is optional type
\end{lstlisting}
The assignment rule is also used in function parameter. A number value can be passed to a function with number type or optional type parameter. An optional value can only be passed to functions with optional type parameter. For example:
\begin{lstlisting}
func add(a, b) {
  return a + b
}

func add_n(a: number, b: number): number {
  return a + b
}

func add_s(a, b: number): string {
  return std::string(a) + std::string(b)
}

func add_s_invalid(a, b:number): string {
  return a + std::string(b)      // invalid; optional propagation
}

func main() {
  var a = add("hello", "world")  // valid
  var b = add(0, 1)  // valid    // invalid
  var c = add_n("hello", "world") // invalid
  var d = add_n(0, 1)             // valid
}
\end{lstlisting}
In the function definition, if there is no keyword representing the type after the function signature, Lucy will treat it as optional typed function. If the function returns a value, it will simply return the result to the caller. If the function does not return any thing, the function will return a none value to the caller. The detailed behavior of the function will be discussed in the function section.

\subsection{Control Flow}
Lucy uses \texttt{if} block and \texttt{while} loop conditional flows. The \texttt{if} statement will first check the value of the condition expression and only if the expression has a boolean value true, the statements inside if block will be executed. The \texttt{else} keyword is used for handling the other cases. Braces around the body are required to clarify the bound of the statements.
\begin{lstlisting}
if expression {
  // statements if expression is true
} else {
  // statements if expression is false
}
\end{lstlisting}
\texttt{if} statement can also take an expression statement before checking the expression's value like the traditional \texttt{for} loop in C language.
\begin{lstlisting}
if var a = add(10, 20); a == 30 {
  // statements if expression is true
} else {
  // statements if expression is false
}
\end{lstlisting}
The \texttt{while} is very similar to the \texttt{if} statement. It checks the expression value every time before executing the statements inside the while loop. Only the condition expression is boolean type and equals true, it will execute the codes inside the loop.
\begin{lstlisting}
while expression {
  // statements if expression is true
}
\end{lstlisting}
Like \texttt{if} statement, the \texttt{while} can take an expression statement before checking the expression's value as well. But this statement will be executed once only.
\begin{lstlisting}
while a = add(10, 20); a < 50 {
  // statements if expression is true
}
\end{lstlisting}
In the above code, \texttt{add(10, 20)} will be evaluated once and in the second iteration, the \texttt{while} statement will only checks the value of \texttt{a < 50}.

\subsection{Function}
Lucy uses \texttt{func} keyword to define a function. In Lucy, every function will returns a value. The keyword representing type can be omitted if the function returns an optional value. If the function returns a none optional value, the keyword representing type must be present.
\begin{lstlisting}
func result() {
  return none
}

func result_n(): number {
  return 0
}
\end{lstlisting}
If the function returns an optional value, then the function does not requires a return statement in every branch. If the return statement is not present, a none value will be returned.
\begin{lstlisting}
func result() {}

func main() {
  var a = result()
  // a is optional type with a none value
}
\end{lstlisting}
But if the function returns a none optional value, a return statement will be required in every branch inside the function.
\begin{lstlisting}
func result():number {
  if true {
    return 0
  } else {
    return 1    // return is required
  }
}

func main() {
  var a = result()
  // a is number type with number value 0
}
\end{lstlisting}
When declaring a function, the keyword representing type of parameter can be omitted if the the parameter is optional type. An optional typed parameter can be assigned with any other typed value but a none optional typed parameter can only be assigned with the corresponding typed value.

\subsection{Parameter Passing}
Like a lot of other languages, invoking a function can be easily done specifying the function name followed by the arguments. The only difference is that every thing in Lucy is passed by value, even if it is an array(As mentioned before, array is a primitive type in Lucy). In this case, modifying an array inside the function will not modify the arguments.
\begin{lstlisting}
func modify(a: []) {
  a = array::append(a, "hello")
  return a
}

func main() {
  var a = ["hello"]
  modify(a)     // this will not modify a
  a = modify(a) // correct to do the modification
}
\end{lstlisting}


\section{Code Organization}
The Lucy program is organized by packages. A Lucy source code file shall start with the package keyword with a string that is package name. For example:
\begin{lstlisting}
package "string"
\end{lstlisting}
The above code defines a \texttt{string} package. An executable Lucy program must contain a main package and the main function must be in the main package.
\begin{lstlisting}
package "main"
func main() {}
\end{lstlisting}
Functions inside package are directly visible to all functions inside this package even if they are in different files. For example we have following codes:
\begin{lstlisting}
// file1.ly
package "string"
func funca() {}

// file2.ly
package "string"
func funcb() {
  funca()
}
\end{lstlisting}
Since the \texttt{file1} and \texttt{file2} are for the same package, when compiling these two files, the Lucy compiler will link these two functions together into one module so that all functions are visible inside the package. The details of how the compiler and linker work will be discussed in the Compilation and Linking section.
\texttt{import} keyword is used to import functions from other packages. for example:
\begin{lstlisting}
package "main"
import "std"
\end{lstlisting}
In the above code imports the \texttt{std} package into the main module. The imported module will be linked during the linking stage by Lucy linker. And to invoke a function inside the \texttt{std} package, Lucy uses \texttt{::} to identify the function.
\begin{lstlisting}
package "main"
import "std"

func main() {
  std::print("Hello World!\n")
}
\end{lstlisting}
When the Lucy compiler compiles the source code, whenever a function is parsed, the function signature will be prefixed by the package name so that in the linking time, the function can be linked by the linker.
\begin{lstlisting}
package "string"

func funca() {} // compiles to string::funca
\end{lstlisting}
So it is fine to define a function with the same name in the imported package. A \texttt{print} function can be defined inside the \texttt{main} package and when invoking \texttt{print} directly, the \texttt{print} function inside main module will be executed and when calling \texttt{std::print}, the \texttt{print} function inside \texttt{std} package will be executed.
\begin{lstlisting}
package "main"
import "std"

func print(val: string) {
  std::print("main: " + val)
}

func main() {
  print("Hello World")      // print main: Hello World
  std::print("Hello World") // print Hello World
}
\end{lstlisting}
The following example shows a way to organize the Lucy codes.
\begin{lstlisting}
game.ly    - package "game"
main.ly    - package "main"
message.ly - package "message"
util.ly    - package "util"
Makefile   - Makefile
\end{lstlisting}
The detailed compilation and linking strategy will be discussed in the Compilation and Linking section.


\section{Syntax Design and Trade-offs}
In this section we will discuss how Lucy syntax and its features improves the the robustness of the program. Lucy is a general purpose language and it is designed to be lightweight. The syntax of Lucy is inspired by Golang and Swift, as well as TypeScript. It tends to be simple, elegant and easy for learning. The goal of the syntax is to provide a way which can miniming the use of tokens without losing the expressiveness of the language. \\
In the first subsection, we will take a look at the syntactic sugar in the Lucy language and discuss how it improves the experience of coding. In the second subsection, we will discuss the advantage and disadvantage of passing by values and passing by references.

\subsection{Syntactic Sugar}
In traditional programming language like C and Java, the basic syntaxes are variable declaration, assignment, expression, if statement and loop statement. In some modern languages, pattern matching will be introduced for replacing the switch statement.

\subsubsection{Variable Declaration}
In the static type programming language, usually the keyword reprsenting the type cannot be omitted so that the compiler can always detect the type of the variable. But in Golang and Swift, the type keyword is suggested to be omitted if the type can be inferred from the expression. For example:
\begin{lstlisting}
var a = 0
\end{lstlisting}
The variable \texttt{a} is a number type variable since the type of 0 is number. Another example is the C language's variable declaration style:
\begin{lstlisting}[language=c]
int a;
\end{lstlisting}
The first style is always suggested in Golang since from the expression, it is easy to know the type of a and the initial value of a is 0. In C language's style, it is easy to know the type of a is int but the initial value is implicit. In fact, a lot of errors are caused by this style since the \texttt{int} variable is not initialized in C. \\
A different example is Python and JavaScript's style. Since they are dynamic typed programming languages there is no need to declare the type of the variable. In Python and JavaScript, the variable declaration is just a simple assignment statement.
\begin{lstlisting}
a = 0
\end{lstlisting}
This style omits the var keyword which actually makes the code a little messy. For example, an editor with syntax highlighter can easily identify the declaration statement using the \texttt{var} style. But comparing with that, the single assignment style cannot be easily identified by human. So considering the readability, a \texttt{var} keyword is required in Lucy for declaring variables.

\subsubsection{If, While and For}
In a traditional language like C, the if statement will take an expression and brackets and braces are required.
\begin{lstlisting}[language=c]
if (expression) {}
else {}
\end{lstlisting}
The Python's style may look a little bit simpler.
\begin{lstlisting}[language=python]
if expression:
else:
\end{lstlisting}
Python uses indentation to identify the statement blocks. This makes the code easier to understand but it may cause problems when moving the code to a new IDE or editor because different editors will use different strategies to handle the indents. \\
In Golang and Swift's style, to improve the robustness of the code, the braces are still required so that even if the indentations are mixed by tabs and spaces, the compiler can still compiles the code correctly. But the bracket can be omitted.
\begin{lstlisting}
if expression {}
else {}
\end{lstlisting}
When we take a look of the \texttt{for} loop in C, the \texttt{for} loop is just a syntactic sugar of the \texttt{while} loop. The \texttt{for} loop is often used since it can take a statement before executing the loop and in each loop, the third statement will be executed at the end of each round. In this case, programmers can simply know the number of the iterations.
\begin{lstlisting}
for(int i=0; i<10; i++)
\end{lstlisting}
Golang and Swift introduce this feature to the \texttt{if} statement. The \texttt{if} statement can take a statement before checking the boolean result of the condition expression.
\begin{lstlisting}
if a = funca(); a > 0 {}
\end{lstlisting}
With this style, it is easy to know where the value of the condition expression comes from. Lucy uses this style in both if and while statement.
Another good example of the \texttt{for} loop is the Python's style. In Python, the \texttt{for} loop is used to iterate a list.
\begin{lstlisting}[language=python]
for item in items:
  print item
\end{lstlisting}
This style omits the indexes in the for loop and it is easier for reading. Lucy takes this style in \texttt{for} loop to iterate arrays, but unlike python, the braces cannot be omitted to keep the robustness of the code.

\subsection{Passing by Values VS Passing by References}
Passing by values and passing by references have been discussed for a long time. In mathematical world, all functions are pure functions which means the parameters of the function will not be changed after executing the function. In C language, everything is passed by value as well, but it can uses pointer to get the address of the parameter and change the value of the parameter. For example:
\begin{lstlisting}
void swap(int* a, int* b) {
  int c = *a;
  *a = *b;
  *b = c;
}
\end{lstlisting}
Since this kind of functions will modify the values of the parameters, for a complex function, it may make the code confusing. Another disadvantage is that this function cannot be optimized for a multi-threads program. But on the other hand, for a pure function, usually it only returns one result, and there is no way to modify two variables. Since this kind of functions can change the values of the parameters, the results can be stored in the parameters, though this may confuse the formal definition of function. An elegant way to solve this problem is to allow function to return more than one results, like Golang and Python:
\begin{lstlisting}[language=python]
a, b = swap(a, b)
\end{lstlisting}
In Lucy, it allows function to return an array.
\begin{lstlisting}
func f() {
  return [true, 0]
}
\end{lstlisting}
Then the caller of this function may use indexes to get the result. In the future version, a syntactic sugar will be introduced to make the language more elegant.
\begin{lstlisting}
func f() {
  return [true, 0]
}

func main() {
  var a, b
  a, b = f()        // syntactic sugar
  var result = f()  // old version
  a = result[0]
  b = result[1]
}
\end{lstlisting}


\section{Compilation and Linking}
As discussed before, Lucy is dynamic typed language with static type check. Lucy compiler is to translate the code to Lucy X assembly language and meanwhile it checks the static type to ensure all type signatures are correct. The Lucy linker is to link different packages according to the import keyword defined in the source code. To compile and link files:
\begin{lstlisting}[language=bash]
lucy -c file1.ly file2.ly file3.ly -o target
\end{lstlisting}
The above command will compile \texttt{file1}, \texttt{file2} and \texttt{file3} and then link them together then output to the target. As mentioned before, a package can be separated into different files. For example, a package is separated into \texttt{file1}, \texttt{file2} and \texttt{file3}, the above command will compile and link them to the target file. It is fine to compile the code in this way:
\begin{lstlisting}[language=bash]
lucy -c file1.ly file2.ly -o package1
lucy -c package1 file2.ly -o package
\end{lstlisting}
But if \texttt{file1}, \texttt{file2} and \texttt{file3} contain two different packages, the above command will throw an exception since two packages cannot be linked into one module. An exception is the \texttt{main} package. Since a Lucy program must contain a main module and main function as its entry point. Linking with \texttt{main} package will make the module executable and change the module's name to \texttt{main}.
\begin{lstlisting}[language=bash]
lucy -c package1_file1.ly -o package1_part1
lucy -c package1_file2.ly -o package1_part2
lucy -c package1_part1 package1_part2 -o package1
lucy -c package1 package2.ly -o package // Error!
lucy -c package1 main.ly -o main        // main is executable
\end{lstlisting}
Separating the linking stage from the compilation stage will fasten the compilation speed. An example makefile of the minesweeper game is given below:
 \begin{lstlisting}[language=bash]
 minesweeper: main.lyo message.lyo game.lyo util.lyo
	lucy -c message.lyo main.lyo game.lyo util.lyo -o minesweeper

main.lyo: main.ly
	lucy -c main.ly -o main.lyo

message.lyo: message.ly
	lucy -c message.ly -o message.lyo

game.lyo:	game.ly
	lucy -c game.ly -o game.lyo

util.lyo: util.ly
	lucy -c util.ly -o util.lyo

clean:
	rm -rf *.lyo
	rm minesweeper
 \end{lstlisting}

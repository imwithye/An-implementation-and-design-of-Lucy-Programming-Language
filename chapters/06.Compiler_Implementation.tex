\chapter{Compiler Implementation}
Lucy is a dynamic language with static type checking. The Lucy compiler compiles Lucy codes to Lucy X assembly codes. It translates the source codes to Lucy X assembly module object and serializes the object to a binary file. The following sections describe the compilation pipeline. In the first stage, the scanner reads in source codes and produces tokens to next stage. In the second section, we will discuss the details of the syntax analyzer. The syntax analyzer produces an abstract syntax tree which will be validated by the semantic analyzer. In the last part of this chapter, we will discuss the implementation of the static type checking system.


\section{Lexical Analysis}
In Lucy's compiler pipeline, lexical analysis is the process of converting a sequence of characters (such as in a computer program or web page) into a sequence of tokens (strings with an identified "meaning"). The Lucy's scanner reads in source codes and recoginizes strings as tokens. The scanner is generated by scanner generator from regular expressions of the tokens. The detailed regular expression's table can be retrieved in the source codes and we will discusses some special cases here. \\
In traditional programming language, statements are separated by semicolons. Lucy uses new line character and semicolon as the statement separator. A special case is the separator in the \texttt{if} and \texttt{while} statement. These two statements may take a statement before checking the value of the condition expression and semicolon is used to separate these two parts. In this case, new line character and semicolons will be recognized as two different tokens.
\begin{lstlisting}[language=java]
[\s\t]+   { /* ignored */ }
[\r\n]+   { return token(Terminals.STMT_TAIL); }
;+        { return token(Terminals.SEMICOLON); }
\end{lstlisting}
Lucy tends to be a simple and modern language. Lucy uses both \texttt{and} and \texttt{\&\&} as its operator so that it can create more readable codes.
\begin{lstlisting}[language=java]
"!"       { return token(Terminals.NOT); }
"not"     { return token(Terminals.NOT); }
"&&"      { return token(Terminals.AND); }
"and"     { return token(Terminals.AND); }
"||"      { return token(Terminals.OR); }
"or"      { return token(Terminals.OR); }
\end{lstlisting}


\section{Syntax Analysis}
Syntactic analysis is the process of analyzing symbols that are produced by the scanner, conforming to the rules of a formal grammar. The Lucy's parser is generated by a LALR parser generator which defines context free grammar and a set of semantic actions.

\subsection{Abstract Syntax Tree}
The abstract syntax tree, or just syntax tree, is a tree representation of the abstract syntactic structure of the Lucy source code. Each node of the tree denotes a construct occurring in the Lucy source code. \\
\subsubsection{Module}
The Lucy abstract syntax tree starts from the \texttt{module} node. Source code inside one package will be compiled into one \texttt{module} and will be translated to \texttt{module} in Lucy X assembly language. The \texttt{module} node is formed by 1. name, 2. imports and 3. functions. The module name is parsed from the \texttt{package} keyword from Lucy source codes. \texttt{imports} and \texttt{functions} are two arrays that hold the imported module name and defined functions. \\
\begin{forest}
[Module
  [Imports[Import[Module Name]][Import[Module Name]][Import[Module Name]]]
  [Functions[Function][Function][...]]
]
\end{forest} \\
\subsubsection{Import}
The \texttt{import} node is nothing more than a string of the imported module name. The imported package will be checked in the linking stage. The details will be described in the Linker Implementation chapter.
\subsubsection{Function}
The \texttt{function} node is formed by 1. name, 2. parameter list, 3. return type, 4. statements. The \texttt{function} node will be transformed into subroutines in the Lucy X assembly language. Parameter and statement list are two arrays that hold the parameter definitions and the statements. \\
\begin{forest}
[Function
  [Name]
  [Parameters[Parameter[Var[Name][Type]]][Parameter[Var[Name][Type]]]]
  [Statements[Statement][Statement][...]]
]
\end{forest} \\
\subsubsection{Parameter}
The \texttt{parameter} node is same as a variable declaration node, which is formed by 1. name and 2. type. The \texttt{parameter} type will only be checked during the semantic analysis time and during the code generation time, the type will be omitted. \\
\begin{forest}
[Parameter
  [VarName]
  [VarType]
]
\end{forest} \\
\subsection{Expression}
In the abstract syntax tree, a Lucy expression is an abstract node which could be 1. Binary Expression, 2. Literal, 3. Var Name and  4. Call expression. The \texttt{Binary Expression} is an abstract node in the abstract syntax tree which is formed by the left expression and right expression. \\
\begin{forest}
[Add Expression
  [VarName [a]]
  [NumberLiteral [10]]
]
\end{forest} \\
The above node is a binary expression node in the abstract syntax tree. The left hand side expression is an expression node of \texttt{Var Name} while the right hand side expression is an expression of \texttt{Number Literal}.
\subsubsection{Statement}
In the abstract syntax tree, a Lucy statement is an abstract node which could be 1. Var Declaration, 2. Assignment, 3. If Else, 4. While, 5. Return and 6. Call statement. \\
The Var Definition statement is formed by a \texttt{var} keyword var name and var type. \\
\begin{forest}
[VarDefinition
  [VarName]
  [VarType]
]
\end{forest} \\
The \texttt{Assignment} statement is formed by \texttt{VarName} and \texttt{Expression}. \\
\begin{forest}
[Assignment
  [VarName]
  [Expression]
]
\end{forest} \\
The \texttt{IfElse} statement is formed by \texttt{Expression} and \texttt{If} statement block and \texttt{Else} statement block. \\
\begin{forest}
[IfElse
  [Expression]
  [IfStatements [Statement][Statement][...]]
  [ElseStatements [Statement][Statement][...]]
]
\end{forest} \\
The \texttt{While} statement is formed by \texttt{Expression} and \texttt{loop} statement block. \\
\begin{forest}
[While
  [Expression]
  [LoopStatements [Statement][Statement][...]]
]
\end{forest} \\
Node that in the Lucy Programming language chapter we discussed the syntactic sugars like the \texttt{for} loop, it is directly translated into while loop by adding the first and last statement. \\
The \texttt{Return} statement is formed by a \texttt{return} keyword and an optional expression. \\
\begin{forest}
[Return
  [Expression]
]
\end{forest} \\
The \texttt{Call} statement is just a function call which invokes the function names. \\
\begin{forest}
[Call
  [FunctionName]
  [Parameters[Parameter][Parameter][...]]
]
\end{forest} \\


\section{Semantic Analysis}
The semantic analyzer checks the variable declaration and the type for each statement. First of all, undeclared variables cannot be referenced in the statement. The variable scope will be limited inside the functions. Second, it checks the type association for each declared variable and expression in the abstract syntax tree. Last but not least, it checks the \texttt{Return} statement in every branch in the function definition.
\subsection{Variable Declaration}
The Lucy code is formed by functions and the function body is formed by an array of statements. The \texttt{Var Declaration} statement is an statement inside the array. For an abstract syntax tree, for example, the tree shown below, \\
\begin{forest}
[Statements
  [VarDeclaration[VarName[a]][Expression[NumberLiteral[13]]]]
  [VarDeclaration[VarName[b]][Expression[NumberLiteral[17]]]]
  [Assignment[VarName[a]][AddExpression[VarName[a]][VarName[b]]]]
]
\end{forest} \\
the \texttt{Assignment} node references two variables \texttt{a} and \texttt{b}. In the variable declaration checking stage, the analyzer looks for the \texttt{Var Declaration} node at the left siblings of the \texttt{Assignment} node. For a complex abstract syntax tree: \\
\begin{forest}
[Statements
  [VarDeclaration[VarName[a]][Expression[NumberLiteral[13]]]]
  [VarDeclaration[VarName[b]][Expression[NumberLiteral[17]]]]
  [IfElse[Condition[EquExpression[VarName[a]][NumberLiteral[13]]]][Statements[[Assignment[VarName[a]][AddExpression[VarName[a]][VarName[b]]]]]]]
]
\end{forest} \\
The \texttt{Assignment} node first looks for the \texttt{Var Declaration} node at the left siblings and if the declaration node cannot be found, it will go to the parent node and look for the \texttt{Var Declaration} node in the siblings of the parent node.
\subsection{Type Checking}
Since Lucy provides a static typing system, the type checking is required in the semantic analysis stage. For each \texttt{Var Declaration} node, the semantic analyzer tries to infer the type of the declared node.
\begin{lstlisting}
var a = 10, b = "hello", c = true, d = a, e = b?, f
\end{lstlisting}
For the example shown above, since the type of \texttt{a} \texttt{Number Literal} node is the number type and the declared variable a is number. So in the above \texttt{Var Declaration} statement, the types of \texttt{a}, \texttt{b}, \texttt{c} and \texttt{d} are fixed. The question mark will make the type to be optional type while still assigning the value to the variable. In this case, the type of variable \texttt{e} is optional and the value of \texttt{e} is "hello". Moreover, if the keyword repersenting the type is not given, it simply assign the value with a \texttt{none} value and an optional type. \\
The semantic analyzer first addresses all \texttt{Var Declaration} node for a \texttt{Var Name} referenced in the expression. And one of the variable is optional type and it simply infers that the type of the expression is optional. If the type is not equal to each other in the expression, a type unmatched exception will be thrown.
\subsection{Return Statement}
If a function returns an optional type value, it can omit the return statement in the function but if a function returns a static typed variable, the return statement must be present in every execution path. Since the function body in Lucy is formed by an array of statements. If the last statement in the array is not a block statement, i.e. \texttt{IfElse} or \texttt{While}, it means that it must be a return statement. Otherwise, the code will either not be executed or return an option type value none. And if the last statement is a block statement, we recursively checks the statement array in the body of the block statements.


\section{Code Generation}
The code generation uses visitor pattern to visit the whole well-formed abstract syntax tree produced by the syntax analyzer. The word well-formed means the abstract tree has been validated by the semantic analyzer. Every single error detected by the semantic analyzer will prevent the code generation stage. \\
To generate a Lucy X module, the generator first visits the \texttt{Module} node and create a module object based on the module name in the syntax tree. Then it visits the \texttt{Imports} node and it pushes them into an \texttt{Imports} array, which will be used in the linking stage to generate the code for the main module. After that, it starts visiting the \texttt{Function} node one by one and generate code for them. \\
\subsection{Function Generation}
For a function node in the abstract syntax tree, it has a parameter list and a function body. Note that the type will be ignored in the code generation stage since all variable in Lucy X assembly is dynamic typed. Because in the Lucy X assembly language, unlike the function in Lucy, the subroutine does not have a parameter list and instead, it uses \texttt{pop} instruction to pop the value from the memory stack and saves them to the parameter variable. So
\begin{lstlisting}
func add(a: number, b: number) {}
\end{lstlisting}
will be translated to
\begin{lstlisting}[language=LucyX]
add {
  pop a
  pop b
}
\end{lstlisting}
After generating the parameter list, the generator starts visiting the statements inside the function body.
\subsection{Statement Generation}
The statement generation will read the statement in the statement array one by one and generate for them. A counter is used to count the number of instructions generated of each statement. And since the target Lucy X code will be in SSA form, i.e. each variable can only be assigned once, temporary registers are used to prevent the reassignment.
\subsubsection{Expression Generation}
For a simple expression, for example the \texttt{Var Name} node and the \texttt{Literal Node} in the syntax tree, the generator simply uses them in the Lucy X instructions. And for a binary expression, the generator first generates the code for the left hand side expression and save the result to a temporary register and then it generates the code for the right hand side expression and saves the result into a temporary register. Finally the generator generates the 3 addtion instructions. So
\begin{lstlisting}
5 + (10 + 11)
\end{lstlisting}
will be translated to
\begin{lstlisting}[language=LucyX]
mv $0 5
add $1 10 11
add $2 $0 $1
\end{lstlisting}
The register starting with \texttt{\$} is the temporary register.
\subsubsection{Var Declaration and Assignment}
In the Lucy X assembly code, there is no \texttt{Var Declaration} instruction. If the register is not defined, the virtual machine will simply create one and assign it with none value. So the \texttt{Var Declaration} node will be directly translated into an \texttt{Assignment} node. \\
To generate an \texttt{Assignment} node, the generator first generates codes for the right hand side expression using the strategy mentioned above and then add a \texttt{mv} instruction after the generation. So
\begin{lstlisting}
a = 5 + (10 + 11)
a = 11
\end{lstlisting}
will be translated to
\begin{lstlisting}[language=LucyX]
mv $0 5
add $1 10 11
add $2 $0 $1
mv $a_$0 $2
mv $a_$1 11
\end{lstlisting}
The register \texttt{\$a\_\$0} and \texttt{\$a\_\$1} are the two images of variable \texttt{a}.
\subsubsection{IfElse and While}
In the assembly level, Lucy X uses \texttt{goto} instruction to represent the \texttt{IfElse} statement in Lucy language. The generator first generates the codes of the condition expression and then it generates the codes for each statement blocks in the \texttt{IfElse} node. A \texttt{goto} instruction will be inserted finally. So
\begin{lstlisting}
func fibonacci(n) {
  if n < 2 {
    return n
  } else {
    return fibonacci(n-1) + fibonacci(n-2)
  }
}
\end{lstlisting}
will be translated to
\begin{lstlisting}[language=LucyX]
fibonacci {
	pop n
	mv $1 n
	mv $2 2
	les $0 $1 $2
	bne $0 9
	mv $3 n
	push $3
	ret
	goto 24
	mv $7 n
	mv $8 1
	sub $6 $7 $8
	push $6
	call fibonacci
	pop $5
	mv $11 n
	mv $12 2
	sub $10 $11 $12
	push $10
	call fibonacci
	pop $9
	add $4 $5 $9
	push $4
	ret
}
\end{lstlisting}
The \texttt{While} loop uses the same strategy as the \texttt{IfElse} block.
\subsubsection{Return}
The \texttt{Return} statement is simply translated into the \texttt{ret} instruction. If the \texttt{Return} statement returns a value or an expression, then the generator first generates the code for the value or expression and then save it to the stack memory and invoke the \texttt{ret} instruction. So
\begin{lstlisting}
return 5 + 10
\end{lstlisting}
will be translated to
\begin{lstlisting}[language=LucyX]
add $0 5 10
push $0
ret
\end{lstlisting}
\subsubsection{Call}
The \texttt{Call} statement is simply translated into the \texttt{call} instruction. Since the subroutine in the Lucy X assembly codes does not have a parameter list, so before invoking the subroutine, the parameter must be pushed into the stack memory. So
\begin{lstlisting}
fibonacci(5)
\end{lstlisting}
will be translated to
\begin{lstlisting}[language=LucyX]
push 5
call fibonacci
\end{lstlisting}
After all statements have been generated, the generator will produce a non-optimized local Lucy X module. To make the module executable, the Lucy Linker will link all the global subroutines and the details will be discussed in the Linker Implementation chapter.

%!TEX root = ../dissertation.tex
\chapter{Sparse Conditional Constant Propagation}

\section{Introduction}
The goal of Constant Propagation is to replace all all constants in the program source code to the defined value and moreover it is always done with constant folding, which evaluate the expression as much as possible. For example, we have following C codes:
\begin{lstlisting}[language=c]
int value() {
  int x = 10;
  int y = x + 10;
  int z = y + x;
  return z;
}
\end{lstlisting}
The value y depends on value x and the value z depends on y. The value of x is known and so we may replace all x by its value before x is modified by some statements. Then the code can be transformed to:
\begin{lstlisting}[language=c]
int value() {
  int x = 10;   // x is dead now since x is not
                // refrenced by any statements any more
  int y = 10 + 10;
  int z = y + 10;
  return z;
}
\end{lstlisting}
By constant folding, the compiler can evaluate the result of y and using constant propagation, y’s value can be propagated to the statements after y before y is modified. So we will have:
\begin{lstlisting}[language=c]
int value() {
  int x = 10;
  int y = 20; // x and y are dead now
  int z = 30;
  return z;
}
\end{lstlisting}
Finally use constant propagation again, we can propagate the z value to the final statement and using the live dead analysis, we can know the values of x, y and z are never used again so we may simply remove them from the code. Finally the code will look like:
\begin{lstlisting}[language=c]
int value() {
  return 30;
}
\end{lstlisting}
Moreover constant propagation will make to some if conditions always be true of false. For example:
\begin{lstlisting}[language=c]
int value() {
  int x = 10;
  int y = x + 10;
  int z = y + x;
  if(z > 20) {
    return 1;
  } else {
    return 0;
  }
}
\end{lstlisting}
After constant propagation the code can be simplified to one statement only. It will make the if statement always be true and the else statement will be redundant here. So the code will be simplified to:
\begin{lstlisting}[language=c]
int value() {
  return 1;
}
\end{lstlisting}


\section{Ideas of Static single assignment form}
The LLVM uses static single assignment to represent its IR. \\
From Wikipedia, the free encyclopedia: \\
In compiler design, static single assignment form (often abbreviated as SSA form or simply SSA) is a property of an intermediate representation (IR), which requires that each variable is assigned exactly once, and every variable is defined before it is used. Existing variables in the original IR are split into versions, new variables typically indicated by the original name with a subscript in textbooks, so that every definition gets its own version. In SSA form, use-def chains are explicit and each contains a single element. \\
The SSA form can be simply achieved by replacing variables with versions. For example:
\begin{lstlisting}[language=c]
a = 0
a = a + 10
b = a + 10
\end{lstlisting}
The code above can be transformed to:
\begin{lstlisting}[language=c]
a = 0
a1 = a + 10
b = a1 + 10
\end{lstlisting}
Each identifier will be at left hand side once and only once. The SSA provides easier way to perform dead and live analysis and usage analysis. But the simple replacement method may fail when applying to conditional block. For example:
\begin{lstlisting}[language=c]
a = 0
if(condition()) {
  a = 10
} else {
  a = 11
}
b = a + 10
\end{lstlisting}
The a’s value depends on the result of condition function. If the function returns true, a will become 10 and if the value of condition function is false, a will become 11. \\
To transform this code to SSA form we will need to introduce the phi operation. The phi operation will choose the value from the executing branch.
\begin{lstlisting}[language=c]
a = 0
if(condition()) {
  a1 = 10
} else {
  a2 = 11
}
a3 = phi(a1, a2)
b = a3 + 10
\end{lstlisting}
In this case, the value of a will be stored inside the variable a3 and the value of a3 will be chosen from a1 and a2 based on the executing branch. \\
The node using phi operation is called PHI Node. Obviously, if there are multi-level if statements, the depth of phi operation will be higher.


\section{The LLVM’s Implementation}
The LLVM optimizer is formed by different passes. The constant propagation optimization is done by sccp pass. The code of sccp pass can be got from \url{http://legup.eecg.utoronto.ca/doxygen/SCCP_8cpp-source.html}.
In LLVM’s implementation, there are 4 different status of a variable.
\begin{enumerate}
\item Undefined: This LLVM Value has no known value yet.
\item Constant: This LLVM Value has a specific constant value.
\item Forcedconstant: This LLVM Value was thought to be undef until ResolvedUndefsIn.  This is treated just like 'constant', but if merged with another (different) constant, it goes to overdefined, instead of asserting.
\item Overdefined: This instruction is not known to be constant, and we know it has a value.
\end{enumerate}
The goal of the optimization is to determine the constant values as many as possible and after that we may replace the constant variables with their values. \\
The LLVM sccp solver is implemented as a instruction visitor and whenever an instruction is visited, a callback function will be invoked to perform analysis or transformation on it. \\
We will discuss some important analysis and transformation method here.

\subsection{visitPHINode}
This method will be invoked whenever a phi node is visited. LLVM applies following rules on this node. \\
If the phi node is a struct node. LLVM will simply mark the node as overdefined. In other words, LLVM won’t do any optimization on struct since the struct value make the optimization much more complex. It may be implemented in later version of LLVM. \\
If the phi node is overdefined, we have to inform all node which uses this phi node, that means those node will be overdefined as well. \\
If the depth of the phi node is very large, it unlike to ever to be marked constant. So we if the depth of phi node is greater than 64, we make this phi node as overdefined. \\
Then we take a look of all operands of the phi node. If any of them are overdefined, the phi node becomes overdefined as well. If they are all constant, and they agree with each other, the phi node becomes the identical constant. If they are constant and don’t agree, the phi node becomes overdefined. If there are no executable operands, the phi remains undefined. For example:
\begin{lstlisting}[language=c]
if(condition()) {
  a1 = b1
} else {
  a2 = b2
}
a3 = phi(a1, a2)
\end{lstlisting}
If one of b1 and b2 is overdefined, or b1 and b2 are constants with different values then the phi node a3 becomes overdefined. Otherwise, b1 and b2 are constants with the same value, the phi node becomes constant.

\subsection{visitBinaryOperator}
Another important method is to visitBinaryOperator. In LLVM’s SSA form, each operation takes exactly 3 variables. So each expression becomes binary expression. We may infer the constant status of the left hand side value based on the right hand side values. LLVM applies the following rules on the binary operation. \\
If the value of first operand is overdefined and value of second operand is also overdefined. then we simply return because it cannot be constant any more. \\
If the value of first operand is constant and value of second operand is also constant, the left hand side value becomes constant. \\
If the value of first operand and value of second operand are not constant or overdefined, it means they are undetermined. We simply return without changing the status of right hand side value. We wait for it to be resolved. \\
Otherwise, one of our operands is overdefined. Try to produce something better than overdefined with some tricks. \\
If this is an AND or OR with 0 or -1, it doesn't matter that the other operand is overdefined since 0 AND anything will be 0 and 1 OR anything will be 1. \\
If both operands are PHI nodes, it is possible that this instruction has a constant value, despite the fact that the PHI node doesn't. Check for this condition now. \\
Since the two PHI nodes are in the same basic block, they must have entries for the same predecessors.  Walk the predecessor list, and if all of the incoming values are constants, and the result of evaluating this expression with all incoming value pairs is the same, then this expression is a constant even though the PHI node is not a constant! For example:
\begin{lstlisting}[language=c]
int a, b;
if(condition()) {
  a = 10;
} else {
  a = 5;
}
if(a>=5) {
  return true;
} else {
  return false;
}
\end{lstlisting}
It looks like the return statement depends on a phi node which makes a become 10 or 5. But in the second conditional block, the expression is always true since the value of a is always greater or equal than 5. Thus, the whole block can be replaced with one return statement. Thus, even though the phi node is overrefined but the node which depends on phi node may be a constant as well. \\
Other LLVM visiting methods do similar things with these two important methods. These methods try to determine as more constant values as possible and for different instructions, and different rules will be applied on them to determine their constant values. \\
The Solve method will process work lists until they are all empty. \\
First we will process the overdefined instruction work list since it drives other things to be overdefined more quickly (Overdefined propagation). In other words, for all values that use an overdefined instruction become overdefined. \\
Second we will process the instruction work list. An instruction is in this work list because it made the transition from undefined to constant. So it will propagate its constant value to its users. For any instructions use the instruction inside the work list, we will notify them to update their constant status of their values. \\
Finally we process the basic block list. We simply notify the basic block that they are newly executable and it will add instructions to the overdefined work list and instruction work list. \\
While solving the data flow for a function, we assume that branches on undefined values cannot reach any of their successors. However, this is not a safe assumption.  After we solve dataflow, the method ResolvedUndefsIn should be use to handle this.  If this returns true, the solver should be rerun. \\
We have discussed the basic ideas of the solver class. The solver class will mark the overdefined value and constant value inside a scope and it will notify instructions that use constant variable with constant value so that it can reduce the value reference. \\
A function pass is a LLVM pass that works against on a LLVM function. Whenever a function is parsed, the function pass will be invoked and the SCCP function pass is a function pass using the SCCPSolver to implement a per-function Sparse Conditional Constant Propagator. \\
When a LLVM function is parsed, it creates a SSCPSolver and mark the first block as executable. Since the function may be invoked by any callee, the parameters of the function must not be the constant type. We simply mark them as overdefined. Then we mark the whole function as undefined and we start call the solver function to solve the constant propagation problem until the ResolvedUndefsIn method return false and there are no undefined instruction in the function. \\
Now each instruction inside the function is solved and the constant status is known. Then we can do live and dead analysis again and if a basic block inside the function is dead, we will remove the basic block from the function. And then after the solver solves the constant type for each instruction inside a basic block, we can replace the constant variable with a constant value by iterating all of the instructions in a live basic block. \\
Moreover, the constant propagation can be implemented on a module. When calling the solver to solve a constant propagation problem of a module, the solve will append the global value to the work list so that the value of the global value can be propagated to the instruction inside the function.

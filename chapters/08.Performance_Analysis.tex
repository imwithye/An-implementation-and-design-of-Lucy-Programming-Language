\chapter{Performance Analysis \& Issues}
Lucy tends to be a simple and general purpose programming language. Unlike the C programming language, Lucy is a virtual machine based programming language and the Lucy compiler translates the Lucy codes to Lucy X assembly codes. There is no directly relation between Lucy and Lucy X. The Lucy is just the high level language representation which can be actually compiled into any other general purpose programming or assembly language. Since Lucy and Lucy X are designed for those who starts learning compiler techniques, due to the limited instruction set, the performance will be much slower than the language which is compiled to machine code. This chapter we will discuss the time complexity and implementation of the Lucy X assembly codes. First we will discuss how the virtual machine performs the instruction and then we will discuss how the code is generated and last of the chapter, we will talk about issues of the Lucy programming language.


\section{Performance Analysis}
\subsection{Implementation of Lucy X and Virtual Machine}
In the Lucy X assembly language, the operand can either be a register or value. If the operand is a register, the virtual machine will first fetch the value from the register and then perform the operation over two values. As we discussed in the virtual machine chapter, the registers are stored in the \texttt{Scope} class. The \texttt{get} method is used to fetch the register from the virtual machine. Usually in the assembly code, the instructions will be directly performed on the CPU's register. But in Lucy's virtual machine, the register first need to be fetched from the \texttt{Scope} class and then map the Java object to the register in the Java virtual machine. Here is an example of the \texttt{add} instruction.
\begin{lstlisting}[language=Java]
// The getValue method will return a value.
// If the operand is a value, then it directly return the value.
// If the operand is a register, then it get the register from
// the virtual machine
// then return the value inside the register.
// vm is the virtual machine instance and the number is the
// position of the operand.
Value v1 = getValue(vm, 1);
Value v2 = getValue(vm, 2);
Register result = getRegister(vm, 0);
result.assignValue(v1.add(v2));
\end{lstlisting}
The above code shows that, to perform an add operation, the virtual machine needs to get the register and then save to the target register. On a real machine, the register is directly addressed by the machine code but in Lucy VM, it requires additional memory operation to get the registers. Assuming the all operation can be done in $O(1)$ time in the \texttt{HashMap} and there is no additional memory operation when Java executing this method, it still requires 3 more instructions to perform the add operation. Here is an example of comparing the Lucy and C's implementation of calculating fibonacci number 20 :
\begin{lstlisting}
$gcc fibonacci.c -o fibonacci
$time ./fibonacci
6765

real 0m0.026s
user 0m0.000s
sys  0m0.000s

$lucy -c fibonacci.ly fibonacci.lyo
$time lucy fibonacci.lyo
6765

real 0m1.083s
user 0m0.015s
sys  0m0.045s
\end{lstlisting}
The code written in C is about more than 40 times faster than Lucy.

\subsection{Code Generation}
To simply the code generation process, the Lucy compiler uses as many temporary registers as possible. For example, when generating the expression node in the abstract tree, Lucy compiler first generates the left and right hand side part recursively and finally generates the operation.
\begin{lstlisting}
sum = (10 + b) + (a + 20)
\end{lstlisting}
will be translated into
\begin{lstlisting}[language=LucyX]
mv $0 b
mv $1 10
add $2 $0 $1
mv $3 a
mv $4 20
add $5 $3 $4
add $6 $2 $5
mv sum $6
\end{lstlisting}
The result will be finally saved in the register \texttt{\$6} and then the \texttt{mv} instruction assigns the result to the \texttt{sum} register. This code is in SSA form and register is only assigned once. The increasing number of registers will lead to bigger memory usage and more redundant instructions. In this case, without any optimization, the speed of Lucy program will be much slower than other languages. \\
A possible solution is to introduce the code optimization stage after linking. For the above example, the constant propagation pass will remove a lot of redundant instructions.
\begin{lstlisting}[language=LucyX]
add $2 b 10
add $5 a 20
add sum $2 $5
\end{lstlisting}
To achieve better memory usage, temporary registers can be folded in the optimization pass.
\begin{lstlisting}[language=LucyX]
add $2 b 10
add sum a 20
add sum $2 sum
\end{lstlisting}

\section{Issues}
As a toy language, or an language designed for learning compiler techniques, the performances issues are not the first consideration. The Lucy compiler generates a Lucy X module which is in SSA form and the Lucy optimizer will optimize the code and remove redundant instructions. In current implementation, the target Lucy X module is not purely in SSA form - it does not have phi node which is required for a SSA assembly code - though it tends to be SSA. And the Lucy optimizer is not implemented yet. \\
Moreover, Lucy VM has no file operation instructions and there are only two system calls. In this case, Lucy VM is not able to manipulate files and monitor system events. A doable solution is adding a special instruction which wrap around the raw Java object. Since in Lucy's VM, every value is a Java object, and the instructions are just an operation to invoke the Java method. Special instruction and object can be introduced to the system so that Lucy can manipulate the Java object directly. This idea will not only solve the file issues but also provide more library to Lucy language. \\
Last important issue is the error handling. Currently Lucy does not provide try and catch block, instead Lucy provides multiple return so that the function can explicitly return the error to the caller. This means programmer must check every errors before handling the return result of a function and the error handling code must be present directly after the code which may cause error. Advantages and disadvantages of these two coding styles are discussed in the Golang's blog.

%!TEX root = ../dissertation.tex
\begin{savequote}[75mm]
Because I want to know how it works.
\qauthor{Jeff Duntemann, Assembly Language Step-by-Step: Programming with Linux}
\end{savequote}

\chapter{Lucy X Assembly Language}
Lucy programming language is compiled and linked to a module in Lucy X assembly language. The Lucy X assembly language will be finally interpreted by Lucy Virtual Machine. This chapter first gives an overview of the Lucy X assembly language and then gives a formal definition of the Lucy X assembly code. Finally we will discuss how the Lucy X assembly code will be organized.

\section{Overview}
The Lucy X assembly language is inspired by MIPS assembly language and it provides dynamic type checking and scope isolation and inheritance. The Lucy X assembly language is formed by modules and routines and it stats executing from the main subroutine.

\subsection{Goal of Lucy X Assembly Language}
Lucy X assembly language(Lucy X) is a general purpose programming language. Unlike other assembly language, Lucy X will not be translated into machine code and instead, it is interpreted by Lucy Virtual Machine. The goal of Lucy X is to provide a general purpose assembly language with as fewer instructions as possible, so that it is much more friendly and easier for the one who starts learning compiler techniques. In this case, unlike other low level programming language, Lucy X does not provide bit operation like shift and bitwise and. Instead, Lucy X uses common mathematical operators like add, sub, multiply, and divide.

\subsection{Design Principles}
Inspired by MIPS and Lisp, Lucy X uses polish notation and all instructions in Lucy X are three address codes. Unlike high level programming language, Lucy X does not allow sub statements. All Lucy X instructions tends to follow the general form $operator$ $result$ $operand1$ $operand2$.

\subsection{Differences between Java Byte Code}
Unlike Java byte code, Lucy X is a dynamic type instruction set. The Lucy virtual machine helps to check the type during the run time. Moreover the Lucy X tends to be as fewer as possible while the Java byte code tends to be as more efficient as possible. Though Lucy X is a lower level programming language like Java byte code, but it does not provide like shift or bit operation. Lucy X provides conditional jump and directly jump instructions to simulate the \texttt{if} and \texttt{while} statement. Moreover, there is no memory operation in Lucy X. The reason of using Lucy X instead of Java byte code is that Lucy X is much easier than Java byte code and it is much easier to understand Lucy's virtual machine than Java byte code virtual machine for the one who starts learning compiler techniques.


\section{Well-Formedness}
It is important to note that this chapter describes ‘well formed’ Lucy X assembly language. There is a difference between what the interpreter accepts and what is considered ‘well formed’. For example, the following instruction is syntactically okay, but not well formed:
\begin{lstlisting}[language=LucyX]
mv $result $result
\end{lstlisting}
because the definition of \texttt{\$result} does not dominate all of its uses. A well formed Lucy X must be syntactically and semantically correct and it must be in single static assignment form.


\section{Instruction Style}
Inspired by MIPS assembly language, Lucy X uses polish notation and all instructions in Lucy X are three address codes. Operands in an instruction can either be a value or a register while the target of the instruction must be a register.
\begin{lstlisting}[language=LucyX]
mv $a 0      // assign 0 to $a
mv $b 1      // assign 1 to $b
add $result $a $b   // operands are registers
add $result $a 2    // operands are register and value
add $result 1 2     // operands are values
\end{lstlisting}
The Lucy X validator will check whether the instruction is valid or not. It checks 1. whether the target is a register; 2. how many operands in the instruction; 3. whether it follows the definition of the instruction. An \texttt{InvalidInstructionException} will be thrown if an invalid instruction is detected.

\subsection{Single Static Assignment Form}
In compiler design, static single assignment form (often abbreviated as SSA form or simply SSA) is a property of an intermediate representation (IR), which requires that each variable is assigned exactly once, and every variable is defined before it is used. Lucy X is not in single static assignment form which means \texttt{add $result $result 1} is allowed in Lucy X's definition. But when we discuss a well formed Lucy X source code, it must be in SSA form so that it can be optimized by the optimizer in compilation time or run time.
\begin{lstlisting}[language=LucyX]
add $result $result 1   // not allowed
add $0 0 0
add $1 0 1
add $result $0 $1       // Single Static Assignment Form
\end{lstlisting}


\section{Identifiers}
Lucy X identifiers come in two basic types: global and local. Global identifiers are subroutines, global variables and local identifiers are the local variables inside a subroutine. The identifier can be any strings which matches regular expression $[\$a-zA-Z]+[a-zA-Z0–9._]*$. The identifier is used to retrieve registers from Lucy virtual machine and the value inside the target register can be either Number, Boolean or String. \\
Examples are shown below:
\begin{lstlisting}[language=LucyX]
; Add two numbers
mv left 10
mv right 20
mv result 0
add result left right

; Sub two numbers
mv diff 0
sub diff left right
\end{lstlisting}
If the identifier is not defined before referenced, the virtual machine will define a new register in the virtual machine and if the identifier has been defined twice, the virtual machine will throw an Overdefined Exception instead. \\
When translating Lucy codes to Lucy X aseembly codes, to improve the readiblity of the assembly codes, the variable name will be translated to register name directly. The temporary registers will be represented by \texttt{\$0}, \texttt{\$1}, \texttt{\$2}...etc.


\section{Scope}
Each subroutine has its own scope in Lucy virtual machine. Subroutines are able to access the global scope and if the identifier can not be retrieved from its own scope, Lucy virtual machine will keep looking for the definition in its parent scope until the module’s root scope. If the identifier does not present in any scope, an \texttt{UndefinedException} will be thrown but only the identifier has been defined in current scope, the \texttt{OverdefinedException} will be thrown. \\
Examples are shown below:
\begin{lstlisting}[language=LucyX]
mv result 0

sum {
  pop a
  pop b
  add result a b
}

main {
  mv a 10
  mv b 12
  push a
  push b
  call sum
  put result
}
\end{lstlisting}
It can be executed correctly since the result identifier is defined at the global scope and it can be retrieved by any subroutines.
An example with \texttt{UndefinedException} and \texttt{OverdefinedException}:
\begin{lstlisting}[language=LucyX]
mv result 0

sum {
  pop a
  pop b
  add result a b
}

sum {}        ; Overdefined

main {
  mv a 10
  mv b 12
  push a
  push b
  call sum
  put result
  call div    ; Undefined
}
\end{lstlisting}


\section{High Level Structure}
\subsection{Module Structure}
Lucy X assembly language is formed by modules. In Lucy programming language, the package will be directly transformed to a Lucy X module. Each Lucy X module has a module name as its identifier and is formed by global instructions and subroutines. A module with main subroutine is an executable module and the main subroutine is its entry point. Inside the module, subroutines are identified by its name and there is no two routines sharing the same name. Modules can be linked together by Lucy linker only if the target module is a main module or the two modules are sharing the same module name, i.e. they are compiled from two different files within the same package.

\subsection{Subroutine Structure}
Lucy X modules are formed by subroutines and the main subroutine will be the entry of the whole program. As a low level assembly code, the subroutine does not provide arguments and the arguments will be retrieved from the stack of the virtual machine. \\
The code transformation is shown below:
\begin{lstlisting}[language=LucyX]
func sum(a, b) {
  return a + b
}

func main() {
  result = sum(10, 10)
}

; The equivalent Lucy X assembly code

sum {
  pop a
  pop b
  add $0 a b
  push $0
  ret
}

main {
  push 10
  push 10
  call sum
  pop result
}
\end{lstlisting}
Before calling a subroutine, the caller shall push the parameters to the stack and invoke the subroutine. The target subroutine will pop the parameter from the stack and save it to the local registers. Then it performs code inside the subroutine. The Return instruction will stop executing the subroutine and return back to the caller, and before the return instruction, the subroutine shall push the result to the stack and the caller can pop it from the stack.
The scope of \texttt{Goto} instruction and Branches instruction are also limited inside the subroutine. All jump instruction jumps to the line of the target instruction but the line number is limited inside the subroutine. For example:
\begin{lstlisting}[language=LucyX]
sub {
  pop a
  pop b
  les cmp a b
  beq cmp 11
  sub result a b
  push result
  ret
  sub result b a
  push result
  ret
}
\end{lstlisting}
The \texttt{BEQ} instruction will be limited inside the \texttt{sub} subroutine so the front end code generator can generate codes by subroutines.

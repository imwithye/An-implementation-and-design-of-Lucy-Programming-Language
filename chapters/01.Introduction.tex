%!TEX root = ../dissertation.tex
\chapter{Introduction}
Programming language is the foundation of the computer world. In the first decades of the 20th century, numerical calculations were based on decimal numbers. Eventually it was realized that logic could be represented with numbers, not only with words. For example, Alonzo Church was able to express the lambda calculus in a formulaic way. The Turing machine was an abstraction of the operation of a tape-marking machine, for example, in use at the telephone companies. At very first, the programming language is just some instructions which are used to drive the computing machine. In 1954, language Fortran was invented at IBM became a widely used high level general purpose programming language. In 1969, the invitation of language C did really change the industry. A widely used operation system Unix is implemented in C and later the open source system kernel Linux and C compiler GCC were introduced to the GNU world. Now they are becoming the most popular and widely used system and compiler.
The programming language does change how people think the problem some how. To understand a language from the compiler is a very good way to understand how the computer and compiler work. Designing a compiler is also a good way explore language features.
This report describes an implementation of the a new language, Lucy programming language. The Lucy programming language is a dynamic language with static type checking. The Lucy compiler compiles Lucy code to Lucy assembly code and the Lucy virtual machine is a Just In Time compiler which translates the Lucy X assembly codes into Java byte code. Lucy aims to be safe, simple and elegant, which is a good start point for those who are interested in compiler design.
This report first discusses the related work in compiler and language design. Second, we will focus on the syntax and design of the Lucy programming language and Lucy X assembly language. Then we will talk about the implementation of the Lucy virtual machine. The implementation of the compiler and linker will be discussed in chapter 6 and chapter 7. Finally we will discuss the performance and issues of Lucy language. Future work and plans will be discuss in the last chapter.


\section{Objective}
Pellentesque habitant morbi tristique senectus et netus et malesuada fames ac turpis egestas. Donec id elementum risus. Duis eu blandit nunc, quis mollis metus. Nam pellentesque, mi vel sodales dignissim, libero eros dictum orci, nec finibus metus enim quis arcu. Aenean tristique luctus ipsum tincidunt sagittis. Nullam sit amet volutpat odio. Aenean congue est elit, ut finibus nunc imperdiet sit amet. Nulla facilisi. Fusce egestas, erat eu accumsan aliquet, leo nunc egestas urna, sed pulvinar risus nibh sed nibh. Proin dui urna, aliquet nec urna sit amet, cursus eleifend sem. Suspendisse feugiat mi eu elit auctor dignissim. Nulla maximus congue purus vel aliquam.


\section{Project Description}
Phasellus in leo magna. Sed vulputate sollicitudin ante quis ultrices. Sed tempus, urna tempor auctor iaculis, dolor nibh aliquam quam, eget viverra lectus nunc at risus. Phasellus elementum lobortis justo sit amet feugiat. Curabitur maximus augue libero, non sollicitudin felis mollis pharetra. Phasellus pellentesque nisi mattis faucibus efficitur. Sed efficitur enim lobortis sapien ornare pharetra. Mauris interdum convallis orci vulputate pulvinar. Pellentesque habitant morbi tristique senectus et netus et malesuada fames ac turpis egestas. Suspendisse gravida facilisis est sed accumsan. Nunc ut libero eget nisi posuere interdum. Sed in sapien tempus, iaculis magna a, sodales nisi. Sed id ipsum est.


\section{Overview of Lucy Lang}
Lorem ipsum dolor sit amet, consectetur adipiscing elit. Etiam quis libero at nibh iaculis ullamcorper a non urna. Nulla sit amet cursus justo, eget vulputate ex. Quisque vulputate eu arcu ornare vehicula. Quisque eros sem, euismod aliquet congue vitae, viverra eu nibh. Cum sociis natoque penatibus et magnis dis parturient montes, nascetur ridiculus mus. Sed venenatis accumsan porta. Pellentesque habitant morbi tristique senectus et netus et malesuada fames ac turpis egestas.


\subsection{Goal of Lucy Programming Language}
Lorem ipsum dolor sit amet, consectetur adipiscing elit. Etiam quis libero at nibh iaculis ullamcorper a non urna. Nulla sit amet cursus justo, eget vulputate ex. Quisque vulputate eu arcu ornare vehicula. Quisque eros sem, euismod aliquet congue vitae, viverra eu nibh. Cum sociis natoque penatibus et magnis dis parturient montes, nascetur ridiculus mus. Sed venenatis accumsan porta. Pellentesque habitant morbi tristique senectus et netus et malesuada fames ac turpis egestas.


\subsection{Lucy Programming Language}
Sed porttitor dui aliquam urna aliquet, id tristique dolor pharetra. Aliquam sed metus ut quam mollis suscipit sit amet a arcu. Mauris elementum felis et ligula cursus volutpat. Integer ipsum urna, facilisis id dignissim sit amet, tempor ut urna. Sed egestas, dui id rutrum facilisis, odio urna congue dolor, eu placerat lorem nisl id mi. Mauris volutpat urna nisi, ac imperdiet orci tincidunt ut. In sed ligula sed neque dapibus consequat id et mauris. Vestibulum id mauris eget urna dignissim sagittis. Fusce auctor enim tellus, et dignissim justo tempus eget. Sed consectetur nec diam a facilisis. Fusce id neque a sem maximus facilisis.

\subsection{Lucy X Assembly Language}
Ut a dignissim est. Maecenas nec ultrices erat. Proin augue sapien, dictum sit amet accumsan nec, porttitor sit amet turpis. In dignissim, neque vitae malesuada facilisis, ex mauris accumsan ligula, ut lobortis risus arcu non nisi. Duis tincidunt imperdiet tellus at rhoncus. Nam non metus nec urna vulputate tristique sed id ex. Integer vitae dignissim ipsum. Donec ac nunc quis elit euismod sodales. Ut commodo aliquet ligula vitae gravida. Curabitur id placerat ante, ac elementum nisl. Vivamus ultricies tincidunt eros, sit amet lacinia justo ornare feugiat.

\subsection{Lucy Virtual Machine}
Sed vestibulum dolor sit amet dolor pharetra ullamcorper. Interdum et malesuada fames ac ante ipsum primis in faucibus. Cum sociis natoque penatibus et magnis dis parturient montes, nascetur ridiculus mus. Nulla egestas scelerisque vulputate. Proin faucibus augue ac nunc aliquam pharetra. Vestibulum ante ipsum primis in faucibus orci luctus et ultrices posuere cubilia Curae; Nunc molestie mauris porttitor, porta lectus ac, congue tellus. Mauris non lorem id nibh sollicitudin venenatis et vel quam. Nunc odio diam, suscipit in molestie id, feugiat auctor turpis.


\section{Organization of the Report}
Pellentesque habitant morbi tristique senectus et netus et malesuada fames ac turpis egestas. Donec id elementum risus. Duis eu blandit nunc, quis mollis metus. Nam pellentesque, mi vel sodales dignissim, libero eros dictum orci, nec finibus metus enim quis arcu. Aenean tristique luctus ipsum tincidunt sagittis. Nullam sit amet volutpat odio. Aenean congue est elit, ut finibus nunc imperdiet sit amet. Nulla facilisi. Fusce egestas, erat eu accumsan aliquet, leo nunc egestas urna, sed pulvinar risus nibh sed nibh. Proin dui urna, aliquet nec urna sit amet, cursus eleifend sem. Suspendisse feugiat mi eu elit auctor dignissim. Nulla maximus congue purus vel aliquam.
